\documentclass[class=report, crop=false, a4paper, 12pt]{standalone}

%Packages import
\usepackage{../pkgs}


\begin{document}
\section{Context}
The outbreak of the COVID-19 pandemic tested the entire world on several levels and changed the concept of what is "normal" thereafter. The devastating health, economic and social consequences that COVID caused, spanned a need to develop novel solutions, for almost every aspect of our lives, that facilitate the adaptation to the new world we're living in. 

\par Educational systems were no exception. In the midst of the pandemic, governments around the world forced institutions to shut down and stop the customary in-person regimen of teaching. By April 2020, most universities transitioned to an adapted remote learning \autocite{winsteadRemoteMicroelectronicsLaboratory2022} that lacked proper support due to the unanticipated nature of the events, leading to new challenges, in particular, the legitimacy of moments of evaluation performed remotely. To counter this problem, different approaches can be taken, namely, changing the method of evaluation, suppressing it altogether \autocite{barronrodriguezRemoteLearningGlobal2021} or, when possible, implement a continuous monitoring solution such as TrustID \red{ref?}. However, there are still unresolved issues that must be addressed in order to implement an end-to-end solution capable of assuring the success of such systems. 

\par One core aspect of them is the face verification task, therefore, the data obtained directly influences the performance. Due to the purpose of the application and expected devices to be used, what is obtained can be classified as from an unconstrained nature. Even though the capture of image is consensual, there is no way of controlling the conditions of capturing the visual data and consequent results. This can be attributed to the fact that it is anticipated that the system will be executed in a laptop or a smartphone, thus the capture device might not be ideal. The more probable input method will be a webcam or the smartphone's front facing camera, so a high variation in pose, resolution, illumination, etc. is not unforeseeable. 

\par Another detail that must be regarded, is the processing power available to execute the system. It is common for the equipments used to have a deficiency of it\footnote{According to the February 2023 Steam hardware survey, roughly 5\% of its users do not have a dedicated GPU.}, which is not suitable for high-demanding applications, as its improved accuracy comes at the cost of increased computational overhead, which can make real-time continuous monitoring unfeasible.

\par In conclusion, the method of choice must take the aforesaid into consideration and be a trade-off between accuracy and computational strain, while also being invariant, to a certain degree, to the posed challenges of capturing the required data. 


\section{Dissertation structure}
\par This dissertation will be divided into different chapters that partitions themselves into sections and subsections. Chapter one relates to the introduction of the dissertation, it will present the context and motivation behind the problem and structure of the document. The document continues to the second chapter, it starts with an overview of the History of AI, carries out a survey about the topic's State-Of-The-Art, presented and summarized through the step-by-step analysis of the pipeline of a Face Recognition system, and ends with a comparison table of the discussed methods. In chapter number three, the implemented methods and experiments are described. The forth chapter will present and discuss the results. Finally, chapter five, will draw conclusions of the work achieved in the past several months and prospects for the future.
\end{document}