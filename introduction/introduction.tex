\documentclass[class=report, crop=false, a4paper, 12pt]{standalone}

%Packages import
\usepackage{../pkgs}


\begin{document}
\section{Context}
The outbreak of the COVID-19 pandemic tested the entire world on several levels and changed the concept of what is considered "normal" thereafter. The devastating health, economic and social consequences that COVID caused, spanned a need to develop novel solutions, for almost every aspect of our lives to facilitate the adaptation to the new world we are now living in. 

\par Educational systems were no exception. In the midst of the pandemic, governments around the world forced institutions to shut down and interrupt the in-person regimen of teaching. By April 2020, most universities transitioned to an adapted remote learning paradigm \autocite{winsteadRemoteMicroelectronicsLaboratory2022} that lacked proper support due to the unanticipated nature of events, leading to new challenges, in particular, the legitimacy of evaluation performed remotely. To counter this problem, different approaches can be taken, namely, changing the method of evaluation, suppressing it altogether \autocite{barronrodriguezRemoteLearningGlobal2021} or, when possible, implement continuous and automated vision-based student monitoring solutions, such as TrustID \autocite{fariaImagebasedFaceVerification2023}. However, there are several unresolved issues that must be addressed in order to implement an end-to-end solution capable of assuring the success of such systems. 

\par One core aspect is the face verification stage, where the visual data obtained from the monitoring system directly influences the rate of success of said stage. Another challenge is the unconstrained nature of the problem due to the purpose of the application and expected devices to be used. There is no way of controlling the conditions of capturing the visual data and consequent results, and the most likely input method will be a webcam or a smartphone's front facing camera, so a high variation in pose, resolution, illumination, etc. is foreseeable. 

\par An additional detail that must be considered is the processing power available to execute the system\footnote{According to the February 2023 Steam hardware survey, roughly 5\% of its users do not have a dedicated GPU.}, since solutions that benefit from higher accuracy comes at the cost of increased computational overhead, which can make real-time continuous monitoring unfeasible.

\par In conclusion, the method of choice must take the aforesaid into consideration and provide a trade-off between accuracy and computational cost, while also being invariant, to a certain degree, to the posed challenges of capturing the required data. Another detail to consider is the accuracy values required in this context. Because the solution is intended to perform image-based student monitoring, the accuracy values do not need to be perfect. Since the monitoring process occurs over a prolonged span of time, there are enough face verifications to supplant possible low accuracy values.

\section{Objectives}
Building \red{upon} the earlier context, the main objective of this dissertation is to evaluate Face Recognition (FR) methods on different benchmarks, compare them to the TrustID project's student monitoring solution and find a better approach. The aim is to identify the model that offers the most favorable balance of performance and computational efficiency, with the notion in mind that in this context a perfect accuracy is not required. To achieve this, the following specific objectives have been established:
\begin{itemize}
    \item Conduct a comprehensive review of state-of-the-art face recognition methods to select the prime ones.
    \item Implement the essential stages of a face recognition pipeline.
    \item Search for a better approach to TrustID's facial recognition module by evaluating the proposed methods on diverse benchmarks.
    \item Fine-tune using relevant datasets in an attempt to further improve the selected approach.
    \item Evaluate the performance of the fine-tuned method using selected benchmarks and discuss the overall best performing solution.
\end{itemize}

\section{Dissertation structure}
\par This dissertation is divided into several chapters. Chapter one relates to the introduction of the dissertation, it presents the context and motivation for this work, and the structure of the document. The document continues in the second chapter, it starts with an overview of the \red{History of Artificial Intelligence (AI)}, carries out a survey about the topic's State-Of-The-Art, presented and summarized through the step-by-step analysis of the pipeline of a Face Recognition system, and ends with a comparison table of the discussed methods. In chapter three, the implemented methods and experiments are described. The forth chapter presents and discusses the results obtained. Finally, chapter five draws conclusions of the work achieved in the past several months and prospects for the future.
\end{document}