\documentclass[class=report, crop=false, a4paper, 12pt]{standalone}

%Packages import
\usepackage{../pkgs}


\begin{document}
\section{History of AI}
% The following sections present a broad overview of the history of Artificial Intelligence (AI) by presenting important articles in order for the reader to be able to have a notion of the progress that has been made over the past decades, the hardships encountered and how important AI is in our lives. 

%Introductory paragraph
% On October 1950, in his article \textit{Computing Machinery and Intelligence}, Alan Turing questioned: "Can machines think?" \autocite{turingCOMPUTINGMACHINERYINTELLIGENCE1950}. At the time, the question was too meaningless to answer since not only the theory but also the technology available weren't developed enough. Nonetheless, Turing still predicted that in the future there would be computers that could, effectively, display human-like intelligence and discernment under the conditions proposed on the aforementioned article.

%Although Turing himself considered this specific inquiry too meaningless to answer, if the problem was viewed as whether a computer could perform adequately in a proposed game called \textit{The Imitation Game}, then machines could effectively be seen as a thought capable system. This game, nowadays also referred to as \textit{The Turing Test}, consists of typewritten questions and answers between 3 operators (two humans and one computer) to determine whom's the machine. Finally, and most importantly, Turing also predicted that in the future there would be computers that could play well said game, i.e, display human-like intelligence and discernment. Approximately seventy-three years have passed since the publication of the aforementioned article and, as predicted by Alan Turing, a number of systems have since passed the \textit{Turing Test} (insert citations here). 

\par The breakthroughs of AI are predominant and its importance in our everyday life is undeniable. The interest in the area grew immensely with all the Turing's theoretical research, the proposal of the first mathematical Artificial Neuron model in 1943 by Warren McCulloch and Walter Pitts~\autocite{mccullochLOGICALCALCULUSIDEAS} or the first successful \gls{ANN} by Belmont Farley and Westley Clark~\autocite{farleySimulationSelforganizingSystems1954}. However, only in 1956, during the \textit{Dartmouth Summer Research Project on Artificial Intelligence} \autocite{mccarthyPROPOSALDARTMOUTHSUMMER}, was the term "Artificial Intelligence" was proposed by John McCarthy \textit{et al.}, beginning what is now considered to be the birth of AI~\autocite{zhangStudyArtificialIntelligence2021}.  

\par The succeeding two decades following the Dartmouth conference were filled with important developments. Namely, the 1959 General Problem Solver implemented by Allen Newel \textit{et al.}~\autocite{newell1959report} or Joseph Weizenbaum's ELIZA (1964), a natural language processing tool~\autocite{weizenbaumELIZAComputerProgram1966}. Unfortunately, part of the interest and development around AI met an unforeseen fade after the 1969 book \textit{The Perceptron: A Probabilistic Model for Information Storage and Organization in the Brain}~\autocite{minsky69perceptrons} that reported the incapability of ANN to solve linear inseparable problems. However, the authors failed to consider other solutions already proposed that solves the linear inseparability, such as the 1965 implementation, by Ivakhnenko and Lapa, of what is considered to be the first deep learning network~\autocite{ivakhnenkoCyberneticPredictingDevices}. Then, an important breakthrough, was achieved in 1979 by Kunihiko Fukushima with the introduction of the first \gls{CNN}. Ten years later, Yann LeCun \textit{et al.} applied for the first time Backpropagation~\autocite{6795724} to a CNN, creating what is now a pillar for most of the modern competition winning networks in computer vision~\autocite{schmidhuberDeepLearningNeural2015}. 
\par The study on Neural Networks continued with special attention for CNNs due to their great performance in image related tasks when compared to others networks~\autocite{lecunGradientBasedLearningApplied1998}. Some relevant examples: in 2003 the MNIST record was broken by Simard \textit{et al.}~\autocite{simardBestPracticesConvolutional2003} and, in 2011, a GPU implementation of a CNN~\autocite{ciresanCommitteeNeuralNetworks2011} achieved superhuman vision performance~\autocite{stallkampManVsComputer2012}. To supplement even more the importance of CNNs and GPUs, only a year later, Alex Krizhevsky \textit{et al.} proposed a Deep CNN trained by GPUs that became the first one of this type to win the ImageNet Large Scale Visual Recognition Challenge (ILSVRC)~\autocite{krizhevskyImageNetClassificationDeep2012}. The year of 2012 was very important for Deep Learning, CNNs and Computer Vision, beginning what is considered to be the start of the new wave of interest in Artificial Intelligence we are currently in. 


\end{document}